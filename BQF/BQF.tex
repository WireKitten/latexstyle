\documentclass[11pt, oneside]{article}   	% use "amsart" instead of "article" for AMSLaTeX format
\usepackage{geometry}                		% See geometry.pdf to learn the layout options. There are lots.
\geometry{letterpaper}                   		% ... or a4paper or a5paper or ... 
%\geometry{landscape}                		% Activate for rotated page geometry
%\usepackage[parfill]{parskip}    		% Activate to begin paragraphs with an empty line rather than an indent
\usepackage{graphicx}				% Use pdf, png, jpg, or eps§ with pdflatex; use eps in DVI mode
								% TeX will automatically convert eps --> pdf in pdflatex	
\usepackage{style}

\newcommand{\coefm}{\left(\begin{matrix} a & \frac{b}{2} \\ \frac{b}{2} & c \end{matrix}\right)}


\title{Binary Quadratic Functions, an introduction}
\author{Kenji Nakagawa}
%\date{}							% Activate to display a given date or no date

\begin{document}
\maketitle
We begin with a few key theorems that we'll see reappear later on.
\begin{theo} \textbf{Arithmetic Progression}
$f(x,y)=ax^2+bxy+cy^2$ has the property that $f(x_1+x_2,y_1+y_2)+f(x_1-x_2,y_1-y_2)=2(f(x_1,y_1)+f(x_2,y_2))$
\end{theo}
The proof is trivial by expansion.


The automorph of a form is a transformation matrix $M$ that satisfies $\col{x'}{y'}=\parray{w}{x}{y}{z} \col{x}{y}$,

\end{document}  